\documentclass[12pt]{article}

%% FONTS
%% To get the default sans serif font in latex, uncomment following line:
 \renewcommand*\familydefault{\sfdefault}
%%
%% to get Arial font as the sans serif font, uncomment following line:
%% \renewcommand{\sfdefault}{phv} % phv is the Arial font
%%
%% to get Helvetica font as the sans serif font, uncomment following line:
% \usepackage{helvet}
\usepackage[small,bf,up]{caption}
\renewcommand{\captionfont}{\footnotesize}
\usepackage[left=1in,right=1in,top=1in,bottom=1in]{geometry}
\usepackage{graphics,epsfig,graphicx,float,subfigure,color}
\usepackage{amsmath,amssymb,amsbsy,amsfonts,amsthm}
\usepackage{url}
\usepackage{boxedminipage}
\usepackage[sf,bf,tiny]{titlesec}
 \usepackage[plainpages=false, colorlinks=true,
   citecolor=blue, filecolor=blue, linkcolor=blue,
   urlcolor=blue]{hyperref}
\usepackage{enumitem}
\usepackage{verbatim}
\usepackage{tikz,pgfplots}

\newcommand{\todo}[1]{\textcolor{red}{#1}}
% see documentation for titlesec package
% \titleformat{\section}{\large \sffamily \bfseries}
\titlelabel{\thetitle.\,\,\,}

\newcommand{\bs}{\boldsymbol}
\newcommand{\alert}[1]{\textcolor{red}{#1}}
\setlength{\emergencystretch}{20pt}

\begin{document}

\begin{center}
  \vspace*{-2cm}
{\small MATH-GA 2012.001 and CSCI-GA 2945.001, Georg Stadler \&
  Dhairya Malhotra (NYU Courant)}
\end{center}
\vspace*{.5cm}
\begin{center}
\large \textbf{%%
Spring 2019: Advanced Topics in Numerical Analysis: \\
High Performance Computing \\
Assignment 3 (due Apr.\ 1, 2019) }
\end{center}



\noindent {\bf Handing in your homework:} Hand in your homework as for
the previous homework assignment (git repo with Makefile), answering
the questions by adding a text or a \LaTeX\ file into your repo.
The git repository \url{https://github.com/NYU-HPC19/homework3.git}
contains the code you can build on for this homework.
\\[.2ex]

% ****************************
\begin{enumerate}
% --------------------------
  \item {\bf Approximating Special Functions Using Taylor Series \& Vectorization.}
    Special functions like trigonometric functions can be expensive to
    evaluate on current processor architectures which are optimized for
    floating-point multiplications and additions. In this assignment, we
    will try to optimize evaluation of $\sin(x)$ for $x\in[-\pi/4,
    \pi/4]$ by replacing the builtin scalar function in C/C++ with a
    vectorized Taylor series approximation,
    \[
      \sin(x) = x - \frac{x^3}{3!} + \frac{x^5}{5!} - \frac{x^7}{7!} + \frac{x^9}{9!} - \frac{x^{11}}{11!} + \cdots
    \]
    The source file \texttt{fast-sin.cpp} in the homework repository
    contains the following functions to evaluate $\{\sin(x_0),
    \sin(x_1), \sin(x_2), \sin(x_3)\}$ for different $x_0,\ldots,x_3$:
    \begin{itemize}
      \item \texttt{sin4\_reference()}: is implemented using the builtin C/C++ function.
      \item \texttt{sin4\_taylor()}: evaluates a truncated Taylor series expansion accurate to about 12-digits.
      \item \texttt{sin4\_intrin()}: evaluates only the first two
        terms in the Taylor series expansion (3-digit accuracy) and
        is vectorized using SSE and AVX intrinsics.
      \item \texttt{sin4\_vec()}: evaluates only the first two terms
        in the Taylor series expansion (3-digit accuracy) and
        is vectorized using the Vec class.
    \end{itemize}
    Your task is to improve the accuracy to 12-digits for {\bf any one}
    vectorized version by adding more terms to the Taylor series
    expansion.  Depending on the instruction set supported by the
    processor you are using, you can choose to do this for either the
    SSE part of the function \texttt{sin4\_intrin()} or the AVX part of
    the function \texttt{sin4\_intrin()} or for the function
    \texttt{sin4\_vec()}.

    {\bf Extra credit:} develop an efficient way to evaluate the
    function outside of the interval $x\in[-\pi/4,\pi/4]$ using
    symmetries.  Explain
    your idea in words and implement it for the function
    \texttt{sin4\_taylor()} and for any one vectorized version.
    Hint: $e^{i \theta} = \cos \theta + i \sin \theta$
    ~~and~~ $e^{i(\theta+\pi/2)} = i e^{i \theta}$.
    % ***************************
  \item {\bf Parallel Scan in OpenMP.} This is an example where the
    shared memory parallel version of an algorithm requires some
    thinking beyond parallelizing for-loops. We aim at parallelizing a
    scan-operation with OpenMP (a serial version is provided in the
    homework repo). Given a (long) vector/array $\bs v\in \mathbb
    R^n$, a scan outputs another vector/array $\bs w\in \mathbb R^n$ of
    the same size with entries
    $$
    w_k = \sum_{i=1}^k v_i \text{ for } k=1,\ldots,n.
    $$
    To parallelize the scan operation with OpenMP using $p$ threads,
    we split the vector into $p$ parts $[v_{k(j)},v_{k(j+1)-1}]$,
    $j=1,\ldots,p$, where $k(1)=1$ and $k(p+1)=n+1$ of (approximately)
    equal length. Now, each thread computes the scan locally and in
    parallel, neglecting the contributions from the other threads.
    Every but the first local scan thus computes results that are off
    by a constant, namely the sums obtains by all the threads with
    lower number.  For instance, all the results obtained by the
    the $r$-th thread are off by
    $$
    \sum_{i=1}^{k(r)-1} v_i = s_1 + \cdots + s_{r-1}
    $$
    which can easily be computed as the sum of the partial sums
    $s_1,\ldots,s_{r-1}$ computed by threads with numbers smaller than
    $r$.  This correction can be done in serial.
    \begin{itemize}
    \item Parallelize the provided serial code. Run it with different
      thread numbers and report the architecture you run it on, the
      number of cores of the processor and the time it takes.
    \end{itemize}
\end{enumerate}


\end{document}
