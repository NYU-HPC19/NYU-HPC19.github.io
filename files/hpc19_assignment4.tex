\documentclass[12pt]{article}

%% FONTS
%% To get the default sans serif font in latex, uncomment following line:
 \renewcommand*\familydefault{\sfdefault}
%%
%% to get Arial font as the sans serif font, uncomment following line:
%% \renewcommand{\sfdefault}{phv} % phv is the Arial font
%%
%% to get Helvetica font as the sans serif font, uncomment following line:
% \usepackage{helvet}
\usepackage[small,bf,up]{caption}
\renewcommand{\captionfont}{\footnotesize}
\usepackage[left=1in,right=1in,top=1in,bottom=1in]{geometry}
\usepackage{graphics,epsfig,graphicx,float,subfigure,color}
\usepackage{amsmath,amssymb,amsbsy,amsfonts,amsthm}
\usepackage{url}
\usepackage{boxedminipage}
\usepackage[sf,bf,tiny]{titlesec}
 \usepackage[plainpages=false, colorlinks=true,
   citecolor=blue, filecolor=blue, linkcolor=blue,
   urlcolor=blue]{hyperref}
\usepackage{enumitem}
\usepackage{verbatim}
\usepackage{tikz,pgfplots}

\newcommand{\todo}[1]{\textcolor{red}{#1}}
% see documentation for titlesec package
% \titleformat{\section}{\large \sffamily \bfseries}
\titlelabel{\thetitle.\,\,\,}

\newcommand{\bs}{\boldsymbol}
\newcommand{\alert}[1]{\textcolor{red}{#1}}
\setlength{\emergencystretch}{20pt}

\begin{document}

\begin{center}
  \vspace*{-2cm}
{\small MATH-GA 2012.001 and CSCI-GA 2945.001, Georg Stadler \&
  Dhairya Malhotra (NYU Courant)}
\end{center}
\vspace*{.5cm}
\begin{center}
\large \textbf{%%
Spring 2019: Advanced Topics in Numerical Analysis: \\
High Performance Computing \\
Assignment 4 (due Apr.\ 15, 2019) }
\end{center}



\noindent {\bf Handing in your homework:} Hand in your homework as for
the previous homework assignments (git repo with Makefile), answering
the questions by adding a text or a \LaTeX\ file to your repo.
\\[.2ex]

% ****************************
\begin{enumerate}
% --------------------------

\item {\bf Matrix-vector operations on a GPU.} Modify the
  CUDA code from class for reduction to implement an inner product between two given
  (long) vectors on a GPU. Then, generalize this code to implement a
  matrix-vector multiplication (no blocking needed here) on the
  GPU. Check the correctness of your implementation by performing the same
  computation on the CPU and compare them. Report the memory band your
  code obtains
  on different GPUs.\footnote{The cuda\{1--5\}.cims.nyu.edu compute servers at
  the Institute have different Nvidia GPUs, for an overview see the list of
  compute servers available at the Institute:
  \url{https://cims.nyu.edu/webapps/content/systems/resources/computeservers}.}
  
\item {\bf 2D Jacobi method on a GPU.}
  Implement the 2D Jacobi method as discussed in the 2nd homework
  assignment using CUDA. The convolution example we'll be showing in
  the class on April 8 will be helpful for that. Monitor the
  performance on different GPUs.\\
  {\em Extra Credit:} Implement the Gauss-Seidel smoothing with
  red-black coloring in CUDA and report the performance.

\item {\bf Pitch your final project.}  Summarize your current plan for
  the final project.  Detail \emph{what} you are planning to do, and
  with \emph{whom} you will be cooperating. The preferred size of
  final project teams is two, but if this makes sense in terms of the
  size of the project, teams of three or doing a project by
  yourself is fine as well.  Each team is expected to give a 10 minute
  presentation about the problem they have worked on and their results
  and experience during the finals' week
  (likely May 20 and 21) and hand in a report as well as their code in
  a repo.  We assume you have
  already talked to us about your project ideas when this homework is due, so a
  short summary is sufficient. We will
  discuss the project expectation during the next class, and are
  available for discussions on April 9, 5-6pm and April 11, 11-12:30
  in office \#1111, and over Slack. We are also posting a list of
  example final projects, but you are encouraged to work on a project that is
  motivated by your own research. Note that we will request
  frequent updates on the progress you are making over the next weeks
  and will help if you get stuck.
\end{enumerate}

\end{document}
